% Options for packages loaded elsewhere
\PassOptionsToPackage{unicode}{hyperref}
\PassOptionsToPackage{hyphens}{url}
%
\documentclass[
]{article}
\usepackage{lmodern}
\usepackage{amssymb,amsmath}
\usepackage{ifxetex,ifluatex}
\ifnum 0\ifxetex 1\fi\ifluatex 1\fi=0 % if pdftex
  \usepackage[T1]{fontenc}
  \usepackage[utf8]{inputenc}
  \usepackage{textcomp} % provide euro and other symbols
\else % if luatex or xetex
  \usepackage{unicode-math}
  \defaultfontfeatures{Scale=MatchLowercase}
  \defaultfontfeatures[\rmfamily]{Ligatures=TeX,Scale=1}
\fi
% Use upquote if available, for straight quotes in verbatim environments
\IfFileExists{upquote.sty}{\usepackage{upquote}}{}
\IfFileExists{microtype.sty}{% use microtype if available
  \usepackage[]{microtype}
  \UseMicrotypeSet[protrusion]{basicmath} % disable protrusion for tt fonts
}{}
\makeatletter
\@ifundefined{KOMAClassName}{% if non-KOMA class
  \IfFileExists{parskip.sty}{%
    \usepackage{parskip}
  }{% else
    \setlength{\parindent}{0pt}
    \setlength{\parskip}{6pt plus 2pt minus 1pt}}
}{% if KOMA class
  \KOMAoptions{parskip=half}}
\makeatother
\usepackage{xcolor}
\IfFileExists{xurl.sty}{\usepackage{xurl}}{} % add URL line breaks if available
\IfFileExists{bookmark.sty}{\usepackage{bookmark}}{\usepackage{hyperref}}
\hypersetup{
  pdftitle={STAT 383 Final Project},
  pdfauthor={Louis Keith},
  hidelinks,
  pdfcreator={LaTeX via pandoc}}
\urlstyle{same} % disable monospaced font for URLs
\usepackage[margin=1in]{geometry}
\usepackage{color}
\usepackage{fancyvrb}
\newcommand{\VerbBar}{|}
\newcommand{\VERB}{\Verb[commandchars=\\\{\}]}
\DefineVerbatimEnvironment{Highlighting}{Verbatim}{commandchars=\\\{\}}
% Add ',fontsize=\small' for more characters per line
\usepackage{framed}
\definecolor{shadecolor}{RGB}{248,248,248}
\newenvironment{Shaded}{\begin{snugshade}}{\end{snugshade}}
\newcommand{\AlertTok}[1]{\textcolor[rgb]{0.94,0.16,0.16}{#1}}
\newcommand{\AnnotationTok}[1]{\textcolor[rgb]{0.56,0.35,0.01}{\textbf{\textit{#1}}}}
\newcommand{\AttributeTok}[1]{\textcolor[rgb]{0.77,0.63,0.00}{#1}}
\newcommand{\BaseNTok}[1]{\textcolor[rgb]{0.00,0.00,0.81}{#1}}
\newcommand{\BuiltInTok}[1]{#1}
\newcommand{\CharTok}[1]{\textcolor[rgb]{0.31,0.60,0.02}{#1}}
\newcommand{\CommentTok}[1]{\textcolor[rgb]{0.56,0.35,0.01}{\textit{#1}}}
\newcommand{\CommentVarTok}[1]{\textcolor[rgb]{0.56,0.35,0.01}{\textbf{\textit{#1}}}}
\newcommand{\ConstantTok}[1]{\textcolor[rgb]{0.00,0.00,0.00}{#1}}
\newcommand{\ControlFlowTok}[1]{\textcolor[rgb]{0.13,0.29,0.53}{\textbf{#1}}}
\newcommand{\DataTypeTok}[1]{\textcolor[rgb]{0.13,0.29,0.53}{#1}}
\newcommand{\DecValTok}[1]{\textcolor[rgb]{0.00,0.00,0.81}{#1}}
\newcommand{\DocumentationTok}[1]{\textcolor[rgb]{0.56,0.35,0.01}{\textbf{\textit{#1}}}}
\newcommand{\ErrorTok}[1]{\textcolor[rgb]{0.64,0.00,0.00}{\textbf{#1}}}
\newcommand{\ExtensionTok}[1]{#1}
\newcommand{\FloatTok}[1]{\textcolor[rgb]{0.00,0.00,0.81}{#1}}
\newcommand{\FunctionTok}[1]{\textcolor[rgb]{0.00,0.00,0.00}{#1}}
\newcommand{\ImportTok}[1]{#1}
\newcommand{\InformationTok}[1]{\textcolor[rgb]{0.56,0.35,0.01}{\textbf{\textit{#1}}}}
\newcommand{\KeywordTok}[1]{\textcolor[rgb]{0.13,0.29,0.53}{\textbf{#1}}}
\newcommand{\NormalTok}[1]{#1}
\newcommand{\OperatorTok}[1]{\textcolor[rgb]{0.81,0.36,0.00}{\textbf{#1}}}
\newcommand{\OtherTok}[1]{\textcolor[rgb]{0.56,0.35,0.01}{#1}}
\newcommand{\PreprocessorTok}[1]{\textcolor[rgb]{0.56,0.35,0.01}{\textit{#1}}}
\newcommand{\RegionMarkerTok}[1]{#1}
\newcommand{\SpecialCharTok}[1]{\textcolor[rgb]{0.00,0.00,0.00}{#1}}
\newcommand{\SpecialStringTok}[1]{\textcolor[rgb]{0.31,0.60,0.02}{#1}}
\newcommand{\StringTok}[1]{\textcolor[rgb]{0.31,0.60,0.02}{#1}}
\newcommand{\VariableTok}[1]{\textcolor[rgb]{0.00,0.00,0.00}{#1}}
\newcommand{\VerbatimStringTok}[1]{\textcolor[rgb]{0.31,0.60,0.02}{#1}}
\newcommand{\WarningTok}[1]{\textcolor[rgb]{0.56,0.35,0.01}{\textbf{\textit{#1}}}}
\usepackage{graphicx,grffile}
\makeatletter
\def\maxwidth{\ifdim\Gin@nat@width>\linewidth\linewidth\else\Gin@nat@width\fi}
\def\maxheight{\ifdim\Gin@nat@height>\textheight\textheight\else\Gin@nat@height\fi}
\makeatother
% Scale images if necessary, so that they will not overflow the page
% margins by default, and it is still possible to overwrite the defaults
% using explicit options in \includegraphics[width, height, ...]{}
\setkeys{Gin}{width=\maxwidth,height=\maxheight,keepaspectratio}
% Set default figure placement to htbp
\makeatletter
\def\fps@figure{htbp}
\makeatother
\setlength{\emergencystretch}{3em} % prevent overfull lines
\providecommand{\tightlist}{%
  \setlength{\itemsep}{0pt}\setlength{\parskip}{0pt}}
\setcounter{secnumdepth}{-\maxdimen} % remove section numbering

\title{STAT 383 Final Project}
\author{Louis Keith}
\date{11/15/2020}

\begin{document}
\maketitle

\hypertarget{introduction}{%
\subsection{Introduction}\label{introduction}}

The purpose of this project is to look at the Coronavirus pandemic in
the state of New Hampshire and compare it to the country as a whole.
There are numerous questions that could be answered by looking at the
data, and we will explore each of them one by one.

\hypertarget{importing-libraries}{%
\subsection{Importing Libraries}\label{importing-libraries}}

The very first thing that needs to be done is importing libraries that
will be useful to this analysis. Readxl will allow importing from excel
sheets and dplyr will provide some functions for tidy data analysis.
Janitor will also provide similar functionality. ggplot2 will be very
helpful for creating plots of all types.

\begin{Shaded}
\begin{Highlighting}[]
\KeywordTok{library}\NormalTok{(readxl)}
\KeywordTok{library}\NormalTok{(dplyr)}
\KeywordTok{library}\NormalTok{(janitor)}
\KeywordTok{library}\NormalTok{(ggplot2)}
\end{Highlighting}
\end{Shaded}

\hypertarget{importing-the-data}{%
\subsection{Importing the Data}\label{importing-the-data}}

Then, we will import the data. Both data sets come from the COVID
tracking project, one has detailed data for just New Hampshire, the
other for the entire country.

\begin{Shaded}
\begin{Highlighting}[]
\NormalTok{new_hampshire_history <-}\StringTok{ }\KeywordTok{read_excel}\NormalTok{(}\StringTok{"C:/Users/louis/Documents/Clarkson/Fall Semester 2020/STAT 383 Probability and Statistics/Final Project/new-hampshire-history.xlsx"}\NormalTok{)}
\NormalTok{national_history <-}\StringTok{ }\KeywordTok{read_excel}\NormalTok{(}\StringTok{"C:/Users/louis/Documents/Clarkson/Fall Semester 2020/STAT 383 Probability and Statistics/Final Project/national-history.xlsx"}\NormalTok{)}
\end{Highlighting}
\end{Shaded}

One thing that was forgotten in the presenting data part of the project
was the total populations of both NH and the country as a whole. Ideally
there would be data for each day, but since we don't have that, current
populations will be stored as constants. This will be for the purpose of
normalizing for population.

\begin{Shaded}
\begin{Highlighting}[]
\NormalTok{nh_pop =}\StringTok{ }\DecValTok{1359711}
\NormalTok{us_pop =}\StringTok{ }\DecValTok{331002651}
\end{Highlighting}
\end{Shaded}

Both populations were pulled from google.

\hypertarget{preliminary-look}{%
\subsection{Preliminary Look}\label{preliminary-look}}

Open both data frames and take a look at the data.

\begin{Shaded}
\begin{Highlighting}[]
\KeywordTok{View}\NormalTok{(new_hampshire_history)}
\KeywordTok{View}\NormalTok{(national_history)}
\end{Highlighting}
\end{Shaded}

Both data sets are formatted similarly. They each contain one row for
each day since the beginning of the data collection began. The data for
New Hampshire begins on March 4th while the national data begins on
January 22nd. A logical first step would be to truncate the data in the
national data set up to March 4th so there is an apples to apples
comparison.

\hypertarget{cleaning}{%
\subsection{Cleaning}\label{cleaning}}

There are also a myriad of columns populated by NA values that we will
get rid of as they are of no use to our analysis. The data quality grade
also isn't useful for us, and for the New Hampshire data it is
superfluous to state it came from NH. Hospitalized and
hospitalized\_cumulative have identical values, so one of them can go
too.

\begin{Shaded}
\begin{Highlighting}[]
\NormalTok{new_hampshire_history =}\StringTok{ }\NormalTok{new_hampshire_history }\OperatorTok\StringTok{ }\KeywordTok{clean_names}\NormalTok{()}
\NormalTok{nhf =}\StringTok{ }\NormalTok{new_hampshire_history }\OperatorTok\StringTok{ }\KeywordTok{select}\NormalTok{(}\OperatorTok{-}\KeywordTok{c}\NormalTok{(data_quality_grade, state, death_confirmed, death_probable, in_icu_currently, negative_tests_antibody, negative_tests_people_antibody, negative_tests_viral, on_ventilator_cumulative, on_ventilator_currently, positive_tests_antibody, positive_tests_antigen, positive_tests_people_antibody, positive_tests_people_antigen, positive_tests_viral, total_test_encounters_viral, total_tests_people_antigen, total_tests_antigen, total_test_encounters_viral_increase, positive_score, hospitalized_cumulative))}
\NormalTok{nhf}\OperatorTok{$}\NormalTok{date <-}\StringTok{ }\KeywordTok{as.Date}\NormalTok{(nhf}\OperatorTok{$}\NormalTok{date)}
\end{Highlighting}
\end{Shaded}

Now that the NH data only contains numeric columns and a properly stored
date column, we can do the same for the national data. This data comes
much cleaner, so the only thing that needs to be done is to convert the
date and clean the names. We do want to only keep the data after March
3rd though.

\begin{Shaded}
\begin{Highlighting}[]
\NormalTok{nf =}\StringTok{ }\NormalTok{national_history }\OperatorTok\StringTok{ }\KeywordTok{clean_names}\NormalTok{()}
\NormalTok{nf}\OperatorTok{$}\NormalTok{date <-}\StringTok{ }\KeywordTok{as.Date}\NormalTok{(nf}\OperatorTok{$}\NormalTok{date)}
\NormalTok{nf =}\StringTok{ }\NormalTok{nf }\OperatorTok\StringTok{ }\KeywordTok{filter}\NormalTok{(date }\OperatorTok{>=}\StringTok{ }\KeywordTok{as.Date}\NormalTok{(}\StringTok{'2020-03-04'}\NormalTok{))}
\end{Highlighting}
\end{Shaded}

At this stage, both data sets should only contain usable dates as well
as numeric columns.

\hypertarget{preliminary-visualizations}{%
\subsection{Preliminary
visualizations}\label{preliminary-visualizations}}

Now that the data cleaning step is done, some exploratory visualizations
can be done to look at the relationships between data. Some of the most
interesting graphs will be deaths over time normalized per million
people.

\begin{Shaded}
\begin{Highlighting}[]
\KeywordTok{ggplot}\NormalTok{(}\DataTypeTok{data =}\NormalTok{ nhf, }\KeywordTok{aes}\NormalTok{(}\DataTypeTok{x =}\NormalTok{ date, }\DataTypeTok{y =}\NormalTok{ (death }\OperatorTok{/}\StringTok{ }\NormalTok{nh_pop) }\OperatorTok{*}\StringTok{ }\DecValTok{1000000}\NormalTok{)) }\OperatorTok{+}\StringTok{ }\KeywordTok{geom_point}\NormalTok{() }\OperatorTok{+}\StringTok{ }\KeywordTok{xlab}\NormalTok{(}\StringTok{"Date"}\NormalTok{) }\OperatorTok{+}\StringTok{ }\KeywordTok{ylab}\NormalTok{(}\StringTok{"Covid Deaths per Million People in New Hampshire"}\NormalTok{) }\OperatorTok{+}\StringTok{ }\KeywordTok{ggtitle}\NormalTok{(}\StringTok{"New Hampshire"}\NormalTok{)}
\end{Highlighting}
\end{Shaded}

\includegraphics{STAT-383-Final-Project_files/figure-latex/death-graphs-1.pdf}

\begin{Shaded}
\begin{Highlighting}[]
\KeywordTok{ggplot}\NormalTok{(}\DataTypeTok{data =}\NormalTok{ nf, }\KeywordTok{aes}\NormalTok{(}\DataTypeTok{x =}\NormalTok{ date, }\DataTypeTok{y =}\NormalTok{ (death }\OperatorTok{/}\StringTok{ }\NormalTok{us_pop) }\OperatorTok{*}\StringTok{ }\DecValTok{1000000}\NormalTok{)) }\OperatorTok{+}\StringTok{ }\KeywordTok{geom_point}\NormalTok{() }\OperatorTok{+}\StringTok{ }\KeywordTok{xlab}\NormalTok{(}\StringTok{"Date"}\NormalTok{) }\OperatorTok{+}\StringTok{ }\KeywordTok{ylab}\NormalTok{(}\StringTok{"Covid Deaths per Million People in the United States"}\NormalTok{) }\OperatorTok{+}\StringTok{ }\KeywordTok{ggtitle}\NormalTok{(}\StringTok{"United States"}\NormalTok{)}
\end{Highlighting}
\end{Shaded}

\includegraphics{STAT-383-Final-Project_files/figure-latex/death-graphs-2.pdf}

Having done absolutely no statistical analysis at this point, the graph
shows that up to November, about 2.5x more people adjusted for
population have died in the United States than in New Hampshire. It also
appears that the graph for the United States appears roughly linear
while the graph for New Hampshire appears to show evidence of
``flattening the curve''. Again, there are no hard conclusions to be
drawn from this, but it suggests that New Hampshire might be handling
the pandemic better than the country as a whole.

We can make the same graphs for the total positive cases so far.

\begin{Shaded}
\begin{Highlighting}[]
\KeywordTok{ggplot}\NormalTok{(}\DataTypeTok{data =}\NormalTok{ nhf, }\KeywordTok{aes}\NormalTok{(}\DataTypeTok{x =}\NormalTok{ date, }\DataTypeTok{y =}\NormalTok{ (positive }\OperatorTok{/}\StringTok{ }\NormalTok{nh_pop) }\OperatorTok{*}\StringTok{ }\DecValTok{1000000}\NormalTok{)) }\OperatorTok{+}\StringTok{ }\KeywordTok{geom_point}\NormalTok{() }\OperatorTok{+}\StringTok{ }\KeywordTok{xlab}\NormalTok{(}\StringTok{"Date"}\NormalTok{) }\OperatorTok{+}\StringTok{ }\KeywordTok{ylab}\NormalTok{(}\StringTok{"Covid Cases per Million People in New Hampshire"}\NormalTok{) }\OperatorTok{+}\StringTok{ }\KeywordTok{ggtitle}\NormalTok{(}\StringTok{"New Hampshire"}\NormalTok{)}
\end{Highlighting}
\end{Shaded}

\includegraphics{STAT-383-Final-Project_files/figure-latex/cases-graph-1.pdf}

\begin{Shaded}
\begin{Highlighting}[]
\KeywordTok{ggplot}\NormalTok{(}\DataTypeTok{data =}\NormalTok{ nf, }\KeywordTok{aes}\NormalTok{(}\DataTypeTok{x =}\NormalTok{ date, }\DataTypeTok{y =}\NormalTok{ (positive }\OperatorTok{/}\StringTok{ }\NormalTok{us_pop) }\OperatorTok{*}\StringTok{ }\DecValTok{1000000}\NormalTok{)) }\OperatorTok{+}\StringTok{ }\KeywordTok{geom_point}\NormalTok{() }\OperatorTok{+}\StringTok{ }\KeywordTok{xlab}\NormalTok{(}\StringTok{"Date"}\NormalTok{) }\OperatorTok{+}\StringTok{ }\KeywordTok{ylab}\NormalTok{(}\StringTok{"Covid Cases per Million People in the United States"}\NormalTok{) }\OperatorTok{+}\StringTok{ }\KeywordTok{ggtitle}\NormalTok{(}\StringTok{"United States"}\NormalTok{)}
\end{Highlighting}
\end{Shaded}

\includegraphics{STAT-383-Final-Project_files/figure-latex/cases-graph-2.pdf}

Similar to the analyses for the previous set of graphs, no hard
conclusions can be drawn from either graph. However, recent reports of
spiking cases can be pretty clearly noted in both graphs. It appears to
be particularly bad in New Hampshire. This will be something that is
explored in a later question.

\hypertarget{question-1-is-new-hampshire-handling-the-pandemic-better-or-worse-than-the-united-states}{%
\subsection{Question 1: Is New Hampshire Handling the Pandemic Better or
Worse Than the United
States?}\label{question-1-is-new-hampshire-handling-the-pandemic-better-or-worse-than-the-united-states}}

Given the preliminary result from above, a one sided test for whether or
not the death rate in New Hampshire adjusted for population is lower
than the same for the United States seems to make sense.

That would make hypotheses: \[H_{0}: p_{deaths-US} - p_{deaths-NH} = 0\]
\[H_{a}: p_{deaths-US} - p_{deaths-NH} > 0\]

We can determine whether or not to reject the null hypothesis by
creating a one sided confidence interval, with the upper bound being 1,
the equation for the lower bound is:

\[p^{hat}_{2} - p^{hat}_{1} - Z_{\alpha}\sqrt{\frac{p^{hat}_{2}(1-p^{hat}_{2})}{n_{2}} + \frac{p^{hat}_{1}(1-p^{hat}_{1})}{n_{1}}}\]
Plugging the values we have on November 7th into this equation, we will
use a 99\% confidence interval:

\begin{Shaded}
\begin{Highlighting}[]
\CommentTok{# the value for death at the top of the table (latest value) divided by the total US population}
\NormalTok{p2 =}\StringTok{ }\NormalTok{nf}\OperatorTok{$}\NormalTok{death[}\DecValTok{1}\NormalTok{]}\OperatorTok{/}\NormalTok{us_pop}
\CommentTok{# doing the same for the NH values. }
\NormalTok{p1 =}\StringTok{ }\NormalTok{nhf}\OperatorTok{$}\NormalTok{death[}\DecValTok{1}\NormalTok{]}\OperatorTok{/}\NormalTok{nh_pop}
\CommentTok{# calculating Z statistic}
\NormalTok{Z =}\StringTok{ }\KeywordTok{qnorm}\NormalTok{(}\FloatTok{0.99}\NormalTok{)}
\NormalTok{lower_bound_proportion_of_deaths =}\StringTok{ }\NormalTok{p2 }\OperatorTok{-}\StringTok{ }\NormalTok{p1 }\OperatorTok{-}\StringTok{ }\NormalTok{Z}\OperatorTok{*}\KeywordTok{sqrt}\NormalTok{((p2}\OperatorTok{*}\NormalTok{(}\DecValTok{1}\OperatorTok{-}\NormalTok{p2)}\OperatorTok{/}\NormalTok{us_pop) }\OperatorTok{+}\StringTok{ }\NormalTok{(p1}\OperatorTok{*}\NormalTok{(}\DecValTok{1}\OperatorTok{-}\NormalTok{p1)}\OperatorTok{/}\NormalTok{nh_pop))}
\end{Highlighting}
\end{Shaded}

The resulting confidence interval is \((0.000294944, 1)\) and does not
contain 0, this provides sufficient evidence to reject the null
hypothesis.

Since we reject the null hypothesis, we conclude that New Hampshire is
likely to have a smaller proportion of deaths than the rest of the
country. This is evidence that the New Hampshire COVID response has
indeed been better than the United States as a whole.

Deaths are not the entire story however. To conclude that New Hampshire
is definitely doing better, it is a good idea to perform the same
analysis for cases in general.

\[H_{0}: p_{cases-US} - p_{cases-NH} = 0\]
\[H_{a}: p_{cases-US} - p_{cases-NH} > 0\] The same exact confidence
interval equation can be used, and the upper bound will still be 1.

\begin{Shaded}
\begin{Highlighting}[]
\CommentTok{# the value for death at the top of the table (latest value) divided by the total US population}
\NormalTok{p2 =}\StringTok{ }\NormalTok{nf}\OperatorTok{$}\NormalTok{positive[}\DecValTok{1}\NormalTok{]}\OperatorTok{/}\NormalTok{us_pop}
\CommentTok{# doing the same for the NH values. }
\NormalTok{p1 =}\StringTok{ }\NormalTok{nhf}\OperatorTok{$}\NormalTok{positive[}\DecValTok{1}\NormalTok{]}\OperatorTok{/}\NormalTok{nh_pop}
\CommentTok{# calculating Z statistic}
\NormalTok{Z =}\StringTok{ }\KeywordTok{qnorm}\NormalTok{(}\FloatTok{0.99}\NormalTok{)}
\NormalTok{lower_bound_proportion_of_cases =}\StringTok{ }\NormalTok{p2 }\OperatorTok{-}\StringTok{ }\NormalTok{p1 }\OperatorTok{-}\StringTok{ }\NormalTok{Z}\OperatorTok{*}\KeywordTok{sqrt}\NormalTok{((p2}\OperatorTok{*}\NormalTok{(}\DecValTok{1}\OperatorTok{-}\NormalTok{p2)}\OperatorTok{/}\NormalTok{us_pop) }\OperatorTok{+}\StringTok{ }\NormalTok{(p1}\OperatorTok{*}\NormalTok{(}\DecValTok{1}\OperatorTok{-}\NormalTok{p1)}\OperatorTok{/}\NormalTok{nh_pop))}
\end{Highlighting}
\end{Shaded}

The resulting confidence interval is \((0.020298, 1)\) and does not
contain 0, this provides sufficient evidence to reject the null
hypothesis.

Since we reject the null hypothesis, we conclude that New Hampshire is
likely to have a smaller proportion of positive cases than the rest of
the country. This too is evidence that the New Hampshire COVID response
has been better than the United States as whole.

Together, positive cases and deaths show that New Hampshire has
responded better than the average within the United States.

\hypertarget{question-2-have-there-been-several-waves-which-wave-are-we-in-are-they-getting-smallerlarger}{%
\subsection{Question 2: Have There Been Several Waves? Which Wave are we
in? Are They Getting
Smaller/Larger?}\label{question-2-have-there-been-several-waves-which-wave-are-we-in-are-they-getting-smallerlarger}}

In order to answer this question, it is a good idea to look at the
positive cases each day.

\begin{Shaded}
\begin{Highlighting}[]
\KeywordTok{ggplot}\NormalTok{(}\DataTypeTok{data =}\NormalTok{ nhf, }\KeywordTok{aes}\NormalTok{(}\DataTypeTok{x =}\NormalTok{ date, }\DataTypeTok{y =}\NormalTok{ positive_increase)) }\OperatorTok{+}\StringTok{ }\KeywordTok{geom_point}\NormalTok{() }\OperatorTok{+}\StringTok{ }\KeywordTok{xlab}\NormalTok{(}\StringTok{"Date"}\NormalTok{) }\OperatorTok{+}\StringTok{ }\KeywordTok{ylab}\NormalTok{(}\StringTok{"New Cases"}\NormalTok{) }\OperatorTok{+}\StringTok{ }\KeywordTok{ggtitle}\NormalTok{(}\StringTok{"New Hampshire"}\NormalTok{)}
\end{Highlighting}
\end{Shaded}

\includegraphics{STAT-383-Final-Project_files/figure-latex/daily-cases-graphs-1.pdf}

\begin{Shaded}
\begin{Highlighting}[]
\KeywordTok{ggplot}\NormalTok{(}\DataTypeTok{data =}\NormalTok{ nf, }\KeywordTok{aes}\NormalTok{(}\DataTypeTok{x =}\NormalTok{ date, }\DataTypeTok{y =}\NormalTok{ positive_increase)) }\OperatorTok{+}\StringTok{ }\KeywordTok{geom_point}\NormalTok{() }\OperatorTok{+}\StringTok{ }\KeywordTok{xlab}\NormalTok{(}\StringTok{"Date"}\NormalTok{) }\OperatorTok{+}\StringTok{ }\KeywordTok{ylab}\NormalTok{(}\StringTok{"New Cases"}\NormalTok{) }\OperatorTok{+}\StringTok{ }\KeywordTok{ggtitle}\NormalTok{(}\StringTok{"United States"}\NormalTok{)}
\end{Highlighting}
\end{Shaded}

\includegraphics{STAT-383-Final-Project_files/figure-latex/daily-cases-graphs-2.pdf}

Looking at the national data, it seems very obvious that there are
three, peaks, and we are in the middle of a third wave. This is
consistent with reports of spiking cases in many states. The New
Hampshire date is also very interesting. It appears that we are in the
middle of a second wave, with the first coinciding mostly with the space
between the first and second waves in the country as a whole. There are
also quite a few days where there are zero cases recorded. This is
probably an error in the data collection.

Looking at the New Hampshire graph, it makes sense that the cases that
were meant to be recorded on the days with zero new cases were lumped
into the next day. This explains why there is a subset of points that
look like they are about twice the height of the rest of the curve. This
is quite unfortunate, but if that assumption is correct, then the
residuals will be worse but the overall fit should remain mostly the
same.

Overall, a tentative conclusion we can draw from this is that the waves
in New Hampshire and the United States do not line up very well. They
have not been synced up, and there have been only two in New Hampshire
while there were three across the nation as a whole.

Hard statistical analysis on this problem is not trivial, so for the
purposes of this project it will not be performed. This is not one of
the three statistical analyses that will be done, but it is very
interesting information nonetheless.

Nonetheless, we can repeat this analysis for the deaths.

\begin{Shaded}
\begin{Highlighting}[]
\KeywordTok{ggplot}\NormalTok{(}\DataTypeTok{data =}\NormalTok{ nhf, }\KeywordTok{aes}\NormalTok{(}\DataTypeTok{x =}\NormalTok{ date, }\DataTypeTok{y =}\NormalTok{ death_increase)) }\OperatorTok{+}\StringTok{ }\KeywordTok{geom_point}\NormalTok{() }\OperatorTok{+}\StringTok{ }\KeywordTok{xlab}\NormalTok{(}\StringTok{"Date"}\NormalTok{) }\OperatorTok{+}\StringTok{ }\KeywordTok{ylab}\NormalTok{(}\StringTok{"New Deaths"}\NormalTok{) }\OperatorTok{+}\StringTok{ }\KeywordTok{ggtitle}\NormalTok{(}\StringTok{"New Hampshire"}\NormalTok{)}
\end{Highlighting}
\end{Shaded}

\includegraphics{STAT-383-Final-Project_files/figure-latex/daily-deaths-graphs-1.pdf}

\begin{Shaded}
\begin{Highlighting}[]
\KeywordTok{ggplot}\NormalTok{(}\DataTypeTok{data =}\NormalTok{ nf, }\KeywordTok{aes}\NormalTok{(}\DataTypeTok{x =}\NormalTok{ date, }\DataTypeTok{y =}\NormalTok{ death_increase)) }\OperatorTok{+}\StringTok{ }\KeywordTok{geom_point}\NormalTok{() }\OperatorTok{+}\StringTok{ }\KeywordTok{xlab}\NormalTok{(}\StringTok{"Date"}\NormalTok{) }\OperatorTok{+}\StringTok{ }\KeywordTok{ylab}\NormalTok{(}\StringTok{"New Deaths"}\NormalTok{) }\OperatorTok{+}\StringTok{ }\KeywordTok{ggtitle}\NormalTok{(}\StringTok{"United States"}\NormalTok{)}
\end{Highlighting}
\end{Shaded}

\includegraphics{STAT-383-Final-Project_files/figure-latex/daily-deaths-graphs-2.pdf}

This appears to roughly support the story that the cases have told. The
United States had a spike just before New Hampshire, and then another
two waves, but the separation appears less than for the cases. The
number of deaths for the second and third waves for the United States
appear to show up less well on the deaths. This could be because
treatments have improved since the early months of the pandemic,
reducing the number of deaths relatives to cases. This will be a subject
of further analysis.

There also appears to be some odd separation between two distinct bands
in the United States data. This could be explained by an error in the
data, but the mechanism is unknown.

The New Hampshire data on its own is more difficult to examine since
there are so few deaths, but it mirrors the same trends that were
observed for the national data. The second wave seems less pronounced in
the death data, potentially because of better treatments.

As with the analysis for cases, this on its own does not provide the
kind of data to make any conclusions. but it may offer grounds for
explanation for things that will be seen later in the analysis.

\hypertarget{question-3-do-cases-in-new-hampshire-correlate-with-cases-in-the-united-states-how-about-deaths}{%
\subsection{Question 3: Do Cases in New Hampshire Correlate with Cases
in The United States? How about
deaths?}\label{question-3-do-cases-in-new-hampshire-correlate-with-cases-in-the-united-states-how-about-deaths}}

This was not in the initial submission for questions, but this is
something that conclusive statistical analysis can be done on, so it
will be explored.

First, we need to create a data frame that contains both columns of data
that we wish to analyze. This can be done by selecting columns from both
of the full data frames available and then joining them by the date
column.

\begin{Shaded}
\begin{Highlighting}[]
\NormalTok{nh_cases_and_deaths =}\StringTok{ }\NormalTok{nhf }\OperatorTok\StringTok{ }
\StringTok{  }\CommentTok{# grab the columns we want from the New Hampshire data and put them in another data frame}
\StringTok{  }\KeywordTok{select}\NormalTok{(date, positive, death) }\OperatorTok
\StringTok{  }\CommentTok{# rename the columns so that they are distinct when joined}
\StringTok{  }\KeywordTok{mutate}\NormalTok{(}\DataTypeTok{nh_positive =}\NormalTok{ positive, }\DataTypeTok{nh_death =}\NormalTok{ death) }\OperatorTok\StringTok{ }
\StringTok{  }\CommentTok{# get rid of the original names for those columns. }
\StringTok{  }\KeywordTok{select}\NormalTok{(}\OperatorTok{-}\KeywordTok{c}\NormalTok{(positive, death))}

\CommentTok{# do the same for the national data}
\NormalTok{us_cases_and_deaths =}\StringTok{ }\NormalTok{nf }\OperatorTok\StringTok{ }
\StringTok{  }\KeywordTok{select}\NormalTok{(date, positive, death) }\OperatorTok
\StringTok{  }\KeywordTok{mutate}\NormalTok{(}\DataTypeTok{us_positive =}\NormalTok{ positive, }\DataTypeTok{us_death =}\NormalTok{ death) }\OperatorTok
\StringTok{  }\KeywordTok{select}\NormalTok{(}\OperatorTok{-}\KeywordTok{c}\NormalTok{(positive, death))}

\CommentTok{# create combined data frame}
\NormalTok{cdf =}\StringTok{ }\KeywordTok{full_join}\NormalTok{(nh_cases_and_deaths, us_cases_and_deaths, }\DataTypeTok{by =} \StringTok{'date'}\NormalTok{)}
\CommentTok{# there are NAs at the beginning of the NH data set for deaths where there should probably be zeros. In order to plot those points we can replace them}
\NormalTok{cdf =}\StringTok{ }\NormalTok{cdf }\OperatorTok\StringTok{ }\KeywordTok{replace}\NormalTok{(., }\KeywordTok{is.na}\NormalTok{(.), }\DecValTok{0}\NormalTok{)}
\end{Highlighting}
\end{Shaded}

Now we can easily create a graph to explore the relationship between
cases in the united states and cases in New Hampshire.

\begin{Shaded}
\begin{Highlighting}[]
\KeywordTok{ggplot}\NormalTok{(}\DataTypeTok{data =}\NormalTok{ cdf, }\KeywordTok{aes}\NormalTok{(}\DataTypeTok{x =}\NormalTok{ us_positive, }\DataTypeTok{y =}\NormalTok{ nh_positive)) }\OperatorTok{+}\StringTok{ }\KeywordTok{geom_point}\NormalTok{() }\OperatorTok{+}\StringTok{ }\KeywordTok{xlab}\NormalTok{(}\StringTok{"US Cases"}\NormalTok{) }\OperatorTok{+}\StringTok{ }\KeywordTok{ylab}\NormalTok{(}\StringTok{"NH Cases"}\NormalTok{) }\OperatorTok{+}\StringTok{ }\KeywordTok{ggtitle}\NormalTok{(}\StringTok{"NH Cases vs US Cases"}\NormalTok{)}
\end{Highlighting}
\end{Shaded}

\includegraphics{STAT-383-Final-Project_files/figure-latex/graph-combined-data-frame-cases-1.pdf}

This is a very interesting plot that tries to predict the number of
cases in NH based on the number of cases in the US. Very interestingly,
it appears that there was a period during which New Hampshire had more
cases than would be suggested by just predicting off of national data.
We can fit a linear model to explore this relationship further.

\[H_{0}: \beta_{1} = 0\] \[H_{0}: \beta_{2} \neq 0\]

\begin{Shaded}
\begin{Highlighting}[]
\NormalTok{nh_cases_vs_us_cases =}\StringTok{ }\KeywordTok{lm}\NormalTok{(}\DataTypeTok{data =}\NormalTok{ cdf, nh_positive }\OperatorTok{~}\StringTok{ }\NormalTok{us_positive)}
\KeywordTok{summary}\NormalTok{(nh_cases_vs_us_cases)}
\end{Highlighting}
\end{Shaded}

\begin{verbatim}
## 
## Call:
## lm(formula = nh_positive ~ us_positive, data = cdf)
## 
## Residuals:
##     Min      1Q  Median      3Q     Max 
## -1355.8  -656.5  -223.0   702.1  1749.7 
## 
## Coefficients:
##              Estimate Std. Error t value Pr(>|t|)    
## (Intercept) 1.357e+03  9.822e+01   13.82   <2e-16 ***
## us_positive 1.070e-03  2.103e-05   50.90   <2e-16 ***
## ---
## Signif. codes:  0 '***' 0.001 '**' 0.01 '*' 0.05 '.' 0.1 ' ' 1
## 
## Residual standard error: 952 on 247 degrees of freedom
## Multiple R-squared:  0.913,  Adjusted R-squared:  0.9126 
## F-statistic:  2591 on 1 and 247 DF,  p-value: < 2.2e-16
\end{verbatim}

From this we can certainly conclude that there is a strong statistical
relationship between these two variables, as is to be expected. Both
p-values for the intercept are about low as R can output, with an
R-squared of 0.913. We can safely reject the null hypothesis. The most
interesting aspect of this graph is the big hump in the data, we can see
what this looks like on the standardized residual plot.

\begin{Shaded}
\begin{Highlighting}[]
\KeywordTok{ggplot}\NormalTok{(}\DataTypeTok{data =}\NormalTok{ cdf, }\KeywordTok{aes}\NormalTok{(}\DataTypeTok{x =}\NormalTok{ us_positive, }\DataTypeTok{y =}\NormalTok{ nh_cases_vs_us_cases}\OperatorTok{$}\NormalTok{residuals}\OperatorTok{/}\KeywordTok{sqrt}\NormalTok{(}\KeywordTok{anova}\NormalTok{(nh_cases_vs_us_cases)}\OperatorTok{$}\StringTok{"Mean Sq"}\NormalTok{[}\DecValTok{2}\NormalTok{]))) }\OperatorTok{+}\StringTok{ }\KeywordTok{geom_point}\NormalTok{() }\OperatorTok{+}\StringTok{ }\KeywordTok{xlab}\NormalTok{(}\StringTok{"US Cases"}\NormalTok{) }\OperatorTok{+}\StringTok{ }\KeywordTok{ylab}\NormalTok{(}\StringTok{"Standardized Residuals"}\NormalTok{) }\OperatorTok{+}\StringTok{ }\KeywordTok{ggtitle}\NormalTok{(}\StringTok{"Standardized Residuals Plot"}\NormalTok{)}
\end{Highlighting}
\end{Shaded}

\includegraphics{STAT-383-Final-Project_files/figure-latex/graph-residuals-cases-1.pdf}

There is an incredibly obvious pattern here, which shows that there is
problem with the linear model. However, not a single data point breaks
the 2 mark on the standardized residuals, which shows that the fit
itself isn't too bad. There is a problem with the shape of the graph
though. Instead of coming up with an alternate relationship, another
explanation can be offered.

From the question 2 analysis we can see that New Hampshire was in the
middle of its first wave while the greater United States was recovering
from its first wave. During this period, The new daily cases was going
down in the United States while spiking in New Hampshire. This offers
explanation for this hump, because it appears to happen around the same
place in the data (May and June). In the residuals, there is a time
immediately after the hump that the cases appear to go below the y = 0
line. This can also be explained by the fact that the United States was
entering its second wave while the New Hampshire was recovering from its
first wave, making the cases in New Hampshire lower than what would be
predicted by the model. Finally, the line moves back above the y = 0
line at the end of the data set because New Hampshire appears to be
spiking right now more severely than the average of the rest of the
nation.

While this is not exactly a desirable result in the residuals, it
combines extremely well with the question 2 analysis to offer a very
good look at what is happening in New Hampshire and how it related to
the United States. It offers statistical backing for the earlier claim
that the waves in New Hampshire are not synchronized with the waves
across the rest of the nation.

We can repeat this analysis for the deaths.

\begin{Shaded}
\begin{Highlighting}[]
\KeywordTok{ggplot}\NormalTok{(}\DataTypeTok{data =}\NormalTok{ cdf, }\KeywordTok{aes}\NormalTok{(}\DataTypeTok{x =}\NormalTok{ us_death, }\DataTypeTok{y =}\NormalTok{ nh_death)) }\OperatorTok{+}\StringTok{ }\KeywordTok{geom_point}\NormalTok{() }\OperatorTok{+}\StringTok{ }\KeywordTok{xlab}\NormalTok{(}\StringTok{"US Deaths"}\NormalTok{) }\OperatorTok{+}\StringTok{ }\KeywordTok{ylab}\NormalTok{(}\StringTok{"NH Deaths"}\NormalTok{) }\OperatorTok{+}\StringTok{ }\KeywordTok{ggtitle}\NormalTok{(}\StringTok{"NH Deaths vs US Deaths"}\NormalTok{)}
\end{Highlighting}
\end{Shaded}

\includegraphics{STAT-383-Final-Project_files/figure-latex/graph-combined-data-frame-deaths-1.pdf}

This is very interesting, as at first glance it does not seem to match
up with the cases data. It does however make a great deal of sense once
one considers that if someone dies of COVID, it is often after a lengthy
weeks-long battle. This explains why the exact same hump appears in the
data but delayed by several weeks. Many of those cases that were
observed in May and June are resulting in deaths that occur later on in
July and August.

We can predict that the residuals will likely tell a similar story, but
shifted to the right. First we must make the model.

\[H_{0}: \beta_{1} = 0\] \[H_{0}: \beta_{2} \neq 0\]

\begin{Shaded}
\begin{Highlighting}[]
\NormalTok{nh_cases_vs_us_cases =}\StringTok{ }\KeywordTok{lm}\NormalTok{(}\DataTypeTok{data =}\NormalTok{ cdf, nh_positive }\OperatorTok{~}\StringTok{ }\NormalTok{us_positive)}
\KeywordTok{summary}\NormalTok{(nh_cases_vs_us_cases)}
\end{Highlighting}
\end{Shaded}

\begin{verbatim}
## 
## Call:
## lm(formula = nh_positive ~ us_positive, data = cdf)
## 
## Residuals:
##     Min      1Q  Median      3Q     Max 
## -1355.8  -656.5  -223.0   702.1  1749.7 
## 
## Coefficients:
##              Estimate Std. Error t value Pr(>|t|)    
## (Intercept) 1.357e+03  9.822e+01   13.82   <2e-16 ***
## us_positive 1.070e-03  2.103e-05   50.90   <2e-16 ***
## ---
## Signif. codes:  0 '***' 0.001 '**' 0.01 '*' 0.05 '.' 0.1 ' ' 1
## 
## Residual standard error: 952 on 247 degrees of freedom
## Multiple R-squared:  0.913,  Adjusted R-squared:  0.9126 
## F-statistic:  2591 on 1 and 247 DF,  p-value: < 2.2e-16
\end{verbatim}

Again, the null hypothesis can be rejected and it can be concluded that
there is a strong relationship. The R-squared is the exact same at
0.913. The p-values are very low. All of this is expected, what will be
more interesting is the residuals.

\begin{Shaded}
\begin{Highlighting}[]
\KeywordTok{ggplot}\NormalTok{(}\DataTypeTok{data =}\NormalTok{ cdf, }\KeywordTok{aes}\NormalTok{(}\DataTypeTok{x =}\NormalTok{ us_death, }\DataTypeTok{y =}\NormalTok{ nh_cases_vs_us_cases}\OperatorTok{$}\NormalTok{residuals}\OperatorTok{/}\KeywordTok{sqrt}\NormalTok{(}\KeywordTok{anova}\NormalTok{(nh_cases_vs_us_cases)}\OperatorTok{$}\StringTok{"Mean Sq"}\NormalTok{[}\DecValTok{2}\NormalTok{]))) }\OperatorTok{+}\StringTok{ }\KeywordTok{geom_point}\NormalTok{() }\OperatorTok{+}\StringTok{ }\KeywordTok{xlab}\NormalTok{(}\StringTok{"US Cases"}\NormalTok{) }\OperatorTok{+}\StringTok{ }\KeywordTok{ylab}\NormalTok{(}\StringTok{"Standardized Residuals"}\NormalTok{) }\OperatorTok{+}\StringTok{ }\KeywordTok{ggtitle}\NormalTok{(}\StringTok{"Standardized Residuals Plot"}\NormalTok{)}
\end{Highlighting}
\end{Shaded}

\includegraphics{STAT-383-Final-Project_files/figure-latex/graph-residuals-deaths-1.pdf}

The result is exactly as predicted after seeing the initial graph. This
tells the same story as the cases graph but shifted a little bit to the
right because the death from any given case is likely to happen several
weeks to a month later.

\hypertarget{question-4-are-treatments-improving-over-time}{%
\subsection{Question 4: Are Treatments Improving Over
Time?}\label{question-4-are-treatments-improving-over-time}}

This is the further analysis for the observation that was made at the
end of the second question. Is the case fatality rate dropping over
time? This was also not a question I initially asked when planning out
this project, but one that arose out of looking at the data.

We can begin this analysis like we have the others, coming up with some
preliminary graphs. We can re-use the cdf data frame from earlier.

\begin{Shaded}
\begin{Highlighting}[]
\KeywordTok{ggplot}\NormalTok{(}\DataTypeTok{data =}\NormalTok{ cdf, }\KeywordTok{aes}\NormalTok{(}\DataTypeTok{x =}\NormalTok{ nh_positive, }\DataTypeTok{y =}\NormalTok{ nh_death)) }\OperatorTok{+}\StringTok{ }\KeywordTok{geom_point}\NormalTok{() }\OperatorTok{+}\StringTok{ }\KeywordTok{xlab}\NormalTok{(}\StringTok{"NH Cases"}\NormalTok{) }\OperatorTok{+}\StringTok{ }\KeywordTok{ylab}\NormalTok{(}\StringTok{"NH Deaths"}\NormalTok{) }\OperatorTok{+}\StringTok{ }\KeywordTok{ggtitle}\NormalTok{(}\StringTok{"New Hampshire"}\NormalTok{)}
\end{Highlighting}
\end{Shaded}

\includegraphics{STAT-383-Final-Project_files/figure-latex/graph-nh-cases-vs-deaths-1.pdf}

This is interesting, it tells a similar story as the graph relating NH
deaths and US deaths. A possible explanation for this is that it
coincides with the majority of deaths from the first wave. Perhaps New
Hampshire hospitals were overwhelmed and unable to give everyone sick
the best treatments that were available. The curve appears to flatten
afterwards relative to before the hump so perhaps treatments improved as
a result. The linear model is the next step.

\[H_{0}: \beta_{1} = 0\] \[H_{0}: \beta_{2} \neq 0\]

\begin{Shaded}
\begin{Highlighting}[]
\NormalTok{nh_model =}\StringTok{ }\KeywordTok{lm}\NormalTok{(}\DataTypeTok{data =}\NormalTok{ cdf, nh_death }\OperatorTok{~}\StringTok{ }\NormalTok{nh_positive)}
\KeywordTok{summary}\NormalTok{(nh_model)}
\end{Highlighting}
\end{Shaded}

\begin{verbatim}
## 
## Call:
## lm(formula = nh_death ~ nh_positive, data = cdf)
## 
## Residuals:
##      Min       1Q   Median       3Q      Max 
## -164.732  -32.345   -7.046   44.267   66.506 
## 
## Coefficients:
##              Estimate Std. Error t value Pr(>|t|)    
## (Intercept) 6.9400327  5.8722199   1.182    0.238    
## nh_positive 0.0528382  0.0009472  55.786   <2e-16 ***
## ---
## Signif. codes:  0 '***' 0.001 '**' 0.01 '*' 0.05 '.' 0.1 ' ' 1
## 
## Residual standard error: 48.03 on 247 degrees of freedom
## Multiple R-squared:  0.9265, Adjusted R-squared:  0.9262 
## F-statistic:  3112 on 1 and 247 DF,  p-value: < 2.2e-16
\end{verbatim}

We can reject the null hypothesis. Interestingly, the intercept is not
significant, but the slope is extremely significant and the R-squared is
very high. The most interesting part of this model is that the slope
that is given is 0.0528, meaning that on average across the whole
pandemic, the case fatality rate has been about 5.28\%.

We can look at the residuals plot.

\begin{Shaded}
\begin{Highlighting}[]
\KeywordTok{ggplot}\NormalTok{(}\DataTypeTok{data =}\NormalTok{ cdf, }\KeywordTok{aes}\NormalTok{(}\DataTypeTok{x =}\NormalTok{ nh_positive, }\DataTypeTok{y =}\NormalTok{ nh_model}\OperatorTok{$}\NormalTok{residuals}\OperatorTok{/}\KeywordTok{sqrt}\NormalTok{(}\KeywordTok{anova}\NormalTok{(nh_model)}\OperatorTok{$}\StringTok{"Mean Sq"}\NormalTok{[}\DecValTok{2}\NormalTok{]))) }\OperatorTok{+}\StringTok{ }\KeywordTok{geom_point}\NormalTok{() }\OperatorTok{+}\StringTok{ }\KeywordTok{xlab}\NormalTok{(}\StringTok{"NH Cases"}\NormalTok{) }\OperatorTok{+}\StringTok{ }\KeywordTok{ylab}\NormalTok{(}\StringTok{"Standardized Residuals"}\NormalTok{) }\OperatorTok{+}\StringTok{ }\KeywordTok{ggtitle}\NormalTok{(}\StringTok{"Standardized Residuals Plot"}\NormalTok{)}
\end{Highlighting}
\end{Shaded}

\includegraphics{STAT-383-Final-Project_files/figure-latex/graph-residuals-nh-1.pdf}

Interesting, while this is a concerning residuals plot, it bodes well
for the conclusions that are being developed thus far. The line (and
thus the y = 0 on this plot) is pulled up by all the deaths that
occurred in the hump, the hump itself is explained above, and the fall
to below -3 can be explained by the aforementioned flattening of the
curve because of treatments developed during the first wave.

The slope of the line puts the case fatality rate in New Hampshire at
about 5.28\% averaged across the pandemic, while the final numbers on
November 7 (489/12241) put the cumulative case fatality rate at about
4\%. This provides further support that perhaps the case fatality rate
is falling over time.

We can and should repeat this analysis for the entire United States.

\begin{Shaded}
\begin{Highlighting}[]
\KeywordTok{ggplot}\NormalTok{(}\DataTypeTok{data =}\NormalTok{ cdf, }\KeywordTok{aes}\NormalTok{(}\DataTypeTok{x =}\NormalTok{ us_positive, }\DataTypeTok{y =}\NormalTok{ us_death)) }\OperatorTok{+}\StringTok{ }\KeywordTok{geom_point}\NormalTok{() }\OperatorTok{+}\StringTok{ }\KeywordTok{xlab}\NormalTok{(}\StringTok{"US Cases"}\NormalTok{) }\OperatorTok{+}\StringTok{ }\KeywordTok{ylab}\NormalTok{(}\StringTok{"US Deaths"}\NormalTok{) }\OperatorTok{+}\StringTok{ }\KeywordTok{ggtitle}\NormalTok{(}\StringTok{"United States"}\NormalTok{)}
\end{Highlighting}
\end{Shaded}

\includegraphics{STAT-383-Final-Project_files/figure-latex/graph-us-cases-vs-deaths-1.pdf}
This graph makes a lot of sense, knowing what we know from previous
analysis, we can explain that the initial hump is due to the first wave
in the May and June months, and a strain on the health care system as
beds fill up. Additionally, the graph is appearing to flatten out as
time goes on, supporting the notion that the case fatality rate is
falling.

\[H_{0}: \beta_{1} = 0\] \[H_{0}: \beta_{2} \neq 0\]

Again, we make the linear model.

\begin{Shaded}
\begin{Highlighting}[]
\NormalTok{us_model =}\StringTok{ }\KeywordTok{lm}\NormalTok{(}\DataTypeTok{data =}\NormalTok{ cdf, us_death }\OperatorTok{~}\StringTok{ }\NormalTok{us_positive)}
\KeywordTok{summary}\NormalTok{(us_model)}
\end{Highlighting}
\end{Shaded}

\begin{verbatim}
## 
## Call:
## lm(formula = us_death ~ us_positive, data = cdf)
## 
## Residuals:
##    Min     1Q Median     3Q    Max 
## -33578 -13312   2642  15224  29911 
## 
## Coefficients:
##              Estimate Std. Error t value Pr(>|t|)    
## (Intercept) 3.272e+04  2.024e+03   16.17   <2e-16 ***
## us_positive 2.350e-02  4.334e-04   54.22   <2e-16 ***
## ---
## Signif. codes:  0 '***' 0.001 '**' 0.01 '*' 0.05 '.' 0.1 ' ' 1
## 
## Residual standard error: 19620 on 247 degrees of freedom
## Multiple R-squared:  0.9225, Adjusted R-squared:  0.9222 
## F-statistic:  2940 on 1 and 247 DF,  p-value: < 2.2e-16
\end{verbatim}

As expected, the null hypothesis can be rejected, the p-values for both
the intercept and slope are incredibly small. The R-squared is again
very high. The most interesting take-away from this is that the case
fatality rate (slope) is 2.35\%, half that of New Hampshire.

We can look at the residuals plot.

\begin{Shaded}
\begin{Highlighting}[]
\KeywordTok{ggplot}\NormalTok{(}\DataTypeTok{data =}\NormalTok{ cdf, }\KeywordTok{aes}\NormalTok{(}\DataTypeTok{x =}\NormalTok{ us_positive, }\DataTypeTok{y =}\NormalTok{ us_model}\OperatorTok{$}\NormalTok{residuals}\OperatorTok{/}\KeywordTok{sqrt}\NormalTok{(}\KeywordTok{anova}\NormalTok{(us_model)}\OperatorTok{$}\StringTok{"Mean Sq"}\NormalTok{[}\DecValTok{2}\NormalTok{]))) }\OperatorTok{+}\StringTok{ }\KeywordTok{geom_point}\NormalTok{() }\OperatorTok{+}\StringTok{ }\KeywordTok{xlab}\NormalTok{(}\StringTok{"US Cases"}\NormalTok{) }\OperatorTok{+}\StringTok{ }\KeywordTok{ylab}\NormalTok{(}\StringTok{"Standardized Residuals"}\NormalTok{) }\OperatorTok{+}\StringTok{ }\KeywordTok{ggtitle}\NormalTok{(}\StringTok{"Standardized Residuals Plot"}\NormalTok{)}
\end{Highlighting}
\end{Shaded}

\includegraphics{STAT-383-Final-Project_files/figure-latex/graph-residuals-us-1.pdf}

This again repeats the same story that was told in NH but to a less
extreme degree. None of the residuals ever get below -2. Interestingly,
the line begins well below y = 0 and flies upwards, this is likely as
hospitals begin to run out of resources in the early phases of the
pandemic. The slow fall later on is likely because treatments improve
over time.

The slope of the line puts the case fatality rate in the United States
at about 2.35\% averaged across the pandemic, while the final numbers on
November 7 (229238/9761481) put the cumulative case fatality rate at
about 2.34\%. This does not provide any evidence that the case fatality
rate has fallen over time. This is very interesting.

\end{document}
